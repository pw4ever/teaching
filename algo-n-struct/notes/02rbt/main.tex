\documentclass[beamer]{standalone}


\begin{document}
\mode* % ignore non-frame text

\frame{\titlepage}

\begin{frame}
  \relax
  {\centering
    quick review\\
  }
\end{frame}

\begin{frame}
  \frametitle{red-black tree}
  \relax
  {\centering
    red-black tree (RBT)\\=\\binary-search tree (BST)\\+\\red-black properties (RBP)\\
  }
\end{frame}

\begin{frame}
\frametitle{red-black properties}
\begin{itemize}
\item node either red or black
\item \alert{root is black}
\item every leaf is black
\item \alert{red node has black children}
\item black height the same
\end{itemize}
\end{frame}

\begin{frame}
  \frametitle{logarithmic height}
  why?
  \begin{itemize}
  \item \alert{well-defined black height} forces a shape of a tree rather than a chain
  \item \alert{abundance of blacks} force the tree height is as much as twice of black height
  \end{itemize}
  this gives good worst-case complexity
\end{frame}

\begin{frame}
  \relax
  {\centering
    the key is to \alert{maintain RBP}\\
  }
\end{frame}

\begin{frame}
  \frametitle{rotation}
  \only<+>{
    {\centering
      \includegraphics[width=0.6\textwidth]{graph/rotate.png}\\
      CLRS3 page 313, Figure 13.2\\
      key: \alert{fix pointers}\\
    }    
  }
  \only<+>{
    {\centering
      \includegraphics[width=0.8\textwidth]{graph/rotate-code.png}\\
      CLRS3 page 313\\     
    }
  }
\end{frame}

\begin{frame}
  \frametitle{node insertion}
  \only<+>{
    \begin{itemize}
    \item first a BST insertion
    \item mark the new node \alert{red}
    \item only ``red-parent-black-children'' and ``black root'' may be violated
    \item \alert{fix them!}
    \item remember to mark the root black
    \end{itemize}
  }
  \only<+>{
    {\centering
      \includegraphics[height=0.8\textheight]{graph/rb-insert-fixup.png}\\
      CLRS3 page 317, Figure 13.4\\
      key: \alert{fix up by cases}\\
    }    
  }
  \only<+>{
    {\centering
      \includegraphics[height=0.8\textheight]{graph/rb-insert-code.png}\\
      CLRS3 page 315\\
    }    
  }
  \only<+>{
    {\centering
      \includegraphics[height=0.8\textheight]{graph/rb-insert-fixup-code.png}\\
      CLRS3 page 316\\
    }    
  }

\end{frame}

\begin{frame}
  \frametitle{node deletion}
  \only<+>{
    \begin{itemize}
    \item  a BST deletion of node $z$, recall the cases:
      \begin{itemize}
      \item 2 ``easy'': $z$ has less than 2 children, $y$ is $z$
      \item 1 ``intermediate'': successor $y$ is $z$'s immediate right child
      \item 1 ``difficult'': successor $y$ is \emph{not} $z$'s immediate right child
      \end{itemize}
    \item keep track of $y$ and $x$ (the node that moves into $y$'s original place)
      \begin{itemize}
      \item beware if $y$ is $z$'s immediate right child
      \end{itemize}
    \item for ``{non-easy},'' color $y$ by $z$'s color
    \item convince yourself why there is no violation if $y$ was red
    \item possible violations if $y$ was black 
      \begin{itemize}
      \item for ``easy,'' $y$/$z$ was root, and a red child has replaced it
      \item if $x$ and new $x.p$ ($y$ for ``intermediate'' and old $y.p$ for ``difficult'') are both red
      \item for ``difficult,'' move $y$ might disturb balanced black height
      \end{itemize}
    \item fix violations
    \end{itemize}
  }

  \only<+>{
    {\centering
      \includegraphics[height=0.8\textheight]{graph/rb-delete-fixup.png}\\
      CLRS3 page 329, Figure 13.7\\
      key: \alert{fix up by cases}\\
    }    
  }
  \only<+>{
    {\centering
      \includegraphics[height=0.3\textheight]{graph/rb-delete-transplant-code.png}\\
      CLRS3 page 323\\
    }    
  } 
  \only<+>{
    {\centering
      \includegraphics[height=0.8\textheight]{graph/rb-delete-code.png}\\
      CLRS3 page 324\\
    }    
  } 
  \only<+>{
    {\centering
      \includegraphics[height=0.8\textheight]{graph/rb-delete-fixup-code.png}\\
      CLRS3 page 326\\
    }    
  } 
\end{frame}

\begin{frame}
  \relax
  {\centering
    so much for the review\\[3ex]
    let us do some exercises\\
  }
\end{frame}

\begin{frame}
  \frametitle{we can use some help in the exercises}
  \relax
  \only<+>{
    {\centering
      \url{https://github.com/pw4ever/pw-teaching-algo-n-struct}\\
      \includegraphics[height=0.8\textheight]{graph/pw-t-as-github-1.png}\quad\includegraphics[height=0.8\textheight]{graph/pw-t-as-github-2.png}\\
    }
  }
  \only<+>{
    \centering
    the setup is quite simple on a Linux VM\\
    \includegraphics[height=0.8\textheight]{graph/setup-example.png}\\
  }
\end{frame}

\begin{frame}
  \frametitle{get your paper and pen}
  \begin{itemize}
  \item start with an empty RBT $T$
  \item shuffle 1 to 18 as an array $A[0..17]$
  \item for $i$ from 0 to 2
    \begin{itemize}
    \item insert $A[6\times i..6\times i+5]$ into $T$
    \item let us pick a node on $T$ and delete it
    \end{itemize}
  \end{itemize}
  trace the whole process, see what is left
\end{frame}

\begin{frame}
  \frametitle{create $T$ and pick $A$}
  \only<+>{
    \centering
    \includegraphics[width=0.6\textwidth]{graph/exer-pick-a.png}\\
  }  
\end{frame}

\begin{frame}
  \frametitle{1st round}
  \only<+>{
    \centering
    insert 15, 14, 18, 9, 6, and 16\\
    \includegraphics[height=0.85\textheight]{graph/exer-1st-insert.png}\\
  }  
  \only<+>{
    \centering
    delete 18\\
    \includegraphics[height=0.5\textheight]{graph/exer-1st-delete.png}\\
  }  
\end{frame}

\begin{frame}
  \frametitle{2nd round}
  \only<+>{
    \centering
    insert 3, 11, 2, 4, 13, and 12\\
    \includegraphics[height=0.85\textheight]{graph/exer-2nd-insert-1.png}\\
  } 
  \only<+>{
    \centering
    insert 3, 11, 2, 4, 13, and 12\\
    \includegraphics[height=0.85\textheight]{graph/exer-2nd-insert-2.png}\\
  }  
  \only<+>{
    \centering
    delete 15\\
    \includegraphics[height=0.85\textheight]{graph/exer-2nd-delete.png}\\
  }  
\end{frame}

\begin{frame}
  \frametitle{3rd round}
  \only<+>{
    \centering
    insert 17, 1, 8, 10, 7, and 5\\
    \includegraphics[height=0.85\textheight]{graph/exer-3rd-insert-1.png}\\
  } 
  \only<+>{
    \centering
    insert 17, 1, 8, 10, 7, and 5\\
    \includegraphics[height=0.85\textheight]{graph/exer-3rd-insert-2.png}\\
  } 
  \only<+>{
    \centering
    insert 17, 1, 8, 10, 7, and 5\\
    \includegraphics[height=0.85\textheight]{graph/exer-3rd-insert-3.png}\\
  }   
  \only<+>{
    \centering
    delete 9\\
    \includegraphics[height=0.85\textheight]{graph/exer-3rd-delete.png}\\
  }  
\end{frame}

\begin{frame}
  \relax
  {\centering
    Q \& A\\
  }
\end{frame}

% \begin{frame}
%   \relax
%   {\centering
%     let us start by reviewing some of the concepts and topics\\
%   }
% \end{frame}




%%%%%%%%%%%%%%%%%%%%%%%%%%%

% \begin{frame}
%   \relax
%   {\centering
%     reference solution\\
%   }
% \end{frame}

%\section{exercises}

% \begin{frame}[label=Q1]
%   \frametitle{Exercise}

%   \begin{myexer}
%   \end{myexer}

%   {\centering
%     \myinternalxref{S1}{Solution}\\
%   }
% \end{frame}

% \begin{frame}[label=A1]
%   \frametitle{Solution to \myinternalxref{Q1}{Exercise}}
%   \begin{proof}
    
%   \end{proof}
% \end{frame}


\end{document}
