\documentclass[beamer]{standalone}


\begin{document}
\mode* % ignore non-frame text

\frame{\titlepage}

\begin{frame}
  we are going to address two things today
  \begin{itemize}
  \item \alert{understand} BST
  \item \alert{how} to write (pseudo)code
    \begin{itemize}
    \item a.k.a., how to \alert{ace} quizzes
    \end{itemize}
  \end{itemize}
\end{frame}

% \begin{frame}
%   \relax
%   {\centering
%     let us start by reviewing some of the concepts and topics\\
%   }
% \end{frame}


\begin{frame}
  \relax
  {\centering
    let us try \alert{understanding BST} first\\
  }
\end{frame}

\begin{frame}
  \frametitle{motivation}
  \only<+-+(1)>{
    \framesubtitle{order}
    
    \alert{which} of the following two ways is faster for understanding the word ``asymptotic?'' \alert{why}?
    \begin{itemize}
    \item look up in a \alert{bound} dictionary.
    \item sift through pages of a dictionary, \alert{unbound and shuffled}.
    \end{itemize}
    \vspace*{2ex}
    {\centering
      please do not cheat by saying ``I googled it''\smiley\\
    }
  }
  \only<+>{
    \relax
    \vspace*{4ex}
    {\centering
      order helps us speed up many tasks, e.g., linear vs. binary search\\
      \alert{order matters!}\\
    }
  }

  \only<+>{
    \framesubtitle{the best of two worlds}
    \begin{itemize}
    \item sorted array
      \begin{itemize}
      \item fast: search ($O(\lg n)$)
      \item slow: insertion/deletion ($O(n)$)
      \end{itemize}
    \item unsorted linked list
      \begin{itemize}
      \item fast: insertion/deletion ($O(1)$\footnote{$O(1)$ for deletion is based on the assumption that the element to be deleted has already been located. otherwise, it is dominated by the $O(n)$ search to locate it.})
      \item slow: search ($O(n)$)
      \end{itemize}
    \end{itemize}
    \vspace*{4ex}
    {\centering
      how to do fast \alert{search} and \alert{insertion/deletion} at the \alert{same time}?\\
    }
  }

  \only<+-+(1)>{
    think: what \alert{structural properties} make \alert{search} and \alert{insertion/deletion} fast?
  }
  \only<+>{
    \relax
    \vspace*{4ex}
    \begin{itemize}
    \item insertion/deletion: pointers
    \item search: the ability to quickly find the median of smaller/larger numbers 
    \end{itemize}
  }
\end{frame}

\begin{frame}
  \frametitle{binary search tree (BST)}
  \only<+>{
    BSTs embody these structural properties
    \begin{itemize}
    \item use \alert{pointers} to organize dynamic structure
    \item point to (hopefully) \alert{median} of smaller/larger numbers
      \begin{itemize}
      \item the catch is that it is \alert{nontrivial} to maintain the ``point-to-median'' property
      \item vanilla BST is at the mercy of insertion/deletion order
      \item self-balancing BSTs\footnote{\url{http://goo.gl/7Kap2q}} (e.g., red-black tree and splay tree) try harder to come closer to the ideal
      \end{itemize}
    \end{itemize}
  }
  \only<+>{
    \framesubtitle{definition}
    BST is a \alert{tree} \alert{plus} the BST \alert{order property}
    \begin{itemize}
    \item \alert{smaller} nodes are on the \alert{left} subtree
    \item \alert{larger} nodes are on the \alert{right} subtree
    \end{itemize}
    what makes BST fast
    \begin{itemize}
    \item operations descend through the tree hierarchy
    \item therefore, complexity is dominated by the \alert{height} of tree
    \item many tree has a height of $O(\lg n)$
    \end{itemize}
    \vspace*{2ex}
    the \alert{recursive} nature of trees makes it natural to process BST \alert{recursively}
  }
  \only<+>{
    \framesubtitle{supported operations and algorithmic solutions}
    \begin{itemize}
      % \footnotesize
    \item walk ($O(n)$): properly schedule visit to \alert{this}, \alert{left}, and \alert{right}
    \item minimum/maximum ($O(\lg n)$\footnote{time complexity for (almost) perfectly balanced BST; skewed BST can degenerate to O(n); same comment applies for following cases}): find leftist/rightest node
    \item search ($O(\lg n)$): go left/right for smaller/larger node
    \item predecessor/successor ($O(\lg n)$): find the maximum/minimum in left/right subtree \emph{or} the \alert{first} smaller/larger ancestor
    \item insert ($O(\lg n)$): search proper position and put in place
    \item delete ($O(\lg n)$): move predecessor or successor here
    \end{itemize}
  }
\end{frame}

\begin{frame}
  \relax
  {\centering
    now, let us see \alert{how} to write (pseudo)code\\
  }
\end{frame}

\begin{frame}
  \frametitle{technique}
  \begin{itemize}
  \item strategy vs tactic
    \begin{itemize}
    \item strategy: program organization and interface specification
    \item tactic: fleshing out details
    \end{itemize}
  \item get big picture right before messing with details
    \begin{itemize}
    \item translate English logical structure to code template
      \begin{itemize}
      \item identifying common patterns: conditionals, iterations, recursions
      \end{itemize}
    \item write code by layers
      \begin{itemize}
      \item example: C programmers write ``\{'' and ``\}'' in pairs
      \end{itemize}
    \end{itemize}
  \item wishful thinking
    \begin{itemize}
    \item treat other procedures (even unfinished) as blackboxs
    \item reveal which procedures need to be written
    \end{itemize}
  \item ``premature optimization is the root of all program evils''
    \begin{itemize}
    \item write simple/clean code first before going for performance
    \item do not rush for iteration if recursion is straightforward
      \begin{itemize}
      \item some recursions can be translated to iterations by simulating stacks
      \item some language compilers can automatically optimize tail recursions to iterations
      \end{itemize}

    \end{itemize}
  \end{itemize}
\end{frame}

\begin{frame}
  \frametitle{to see is to believe}
  \begin{itemize}
  \item C is not a good language for learning algorithm
    \begin{itemize}
    \item it forces you to think at a relatively low abstraction level
    \item you have to fiddle with memory (de)allocation if you go beyond static data structures
    \item it is more like sugar-coated assembly
    \item nevertheless, it is the status quo in system programming and you have no other choice if you hack\footnote{\url{http://goo.gl/6pazHs}} Linux kernel
    \end{itemize}
  \item \alert{Python} is a popular alternative; \alert{Scheme} is an excellent choice
  \item I am going to show you how I code BST in \alert{Common Lisp}
  \item if you do not speak Common Lisp, treat it as a kind of \alert{executable pseudo-code}
  \item the point is to demonstrate the \alert{process} of coding
  \end{itemize}
\end{frame}

\begin{frame}
  \frametitle{caveat emptor}
  \begin{quote}
    Thus, programs must be written for people to read, and only incidentally for machines to execute.\\
    \flushright Abelson, Sussman \& Sussman, \emph{SICP}\footnote{\url{http://goo.gl/O4Iq1P}}
  \end{quote}
  \vspace*{4ex}

  \begin{itemize}
  \item the code you are about to see are optimized for readability/simplicity, not performance
  \item you do not have to memorize the code, it is easier to reinvent the code than to memorize it \alert{once you understand it}
  \item nevertheless, follow the procedure once or twice just for the experience
  \end{itemize}
\end{frame}

\begin{frame}
  \frametitle{know what to do}
  \begin{itemize}
  \item nodes in BST have:
    \begin{itemize}
    \item \alert{value}
    \item \alert{left} and \alert{right children}
    \item (optional) \alert{parent}
    \end{itemize}
  \item BST encodes order, so can be used for querying:
    \begin{itemize}
    \item  \alert{minimum} and \alert{maximum}
    \item \alert{predecessor} and \alert{successor}
    \end{itemize}
  \item constructing/manipulating BST require:
    \begin{itemize}
    \item \alert{inserting} a node
    \item \alert{deleting} a node
    \end{itemize}
    that \alert{preserve BST order property}
  \item \alert{search} a node with a given value
  \end{itemize}
\end{frame}

\begin{frame}
  \relax
  {\centering
    let us get started now\\
  }
\end{frame}



\begin{frame}
  \frametitle{running example}
  {\centering
    \includegraphics[width=0.7\textwidth]{example.png}\\
    CLRS3 Figure 12.2 on Page 209\\
  }
\end{frame}

\begin{frame}[label=Q2]
  \frametitle{Exercise}

  \begin{myexer}
    given a BST (e.g., the running example), find \alert{an} order of insertions that construct that tree
    {\centering
      \includegraphics[width=0.7\textwidth]{example.png}\\
    }
  \end{myexer}

  {\centering
    \myinternalxref{A2}{Solution}\\
  }
\end{frame}

\begin{frame}
  \frametitle{auxiliaries}
  \only<+>{
    \framesubtitle{definition}
    {\centering
      \includegraphics[width=0.7\textwidth]{bst-node-defstruct.png}\\
      the hairy stuff in the middle is for ``pretty printing'' the structure\\
    }
  }
  \only<+>{
    \framesubtitle{pretty printing}
    {\centering
      \includegraphics[width=0.5\textwidth]{example.png}\\
      \includegraphics[width=0.5\textwidth]{bst-node-pprint.png}\\
    }
  }
  \only<+>{
    \framesubtitle{build BST from values}
    {\centering
      \includegraphics[width=0.7\textwidth]{build-bst-from-values.png}\\
    }
  }
\end{frame}

\begin{frame}
  \frametitle{preorder, inorder, and postorder walk}
  {\centering
    \includegraphics[width=0.7\textwidth]{bst-walk.png}\\
  }
\end{frame}

\begin{frame}[label=Q1]
  \frametitle{Exercise}

  \begin{myexer}
    which one among preorder, inorder, and postorder walk is equivalent to sorting?
  \end{myexer}

  {\centering
    \myinternalxref{A1}{Solution}\\
  }
\end{frame}

\begin{frame}
  \frametitle{minimum/maximum and predecessor/successor}
  \only<+>{
    {\centering
      \includegraphics[width=0.7\textwidth]{bst-minmax-predsucc.png}\\
    }
  }
  \only<+>{
    {\centering
      \includegraphics[width=0.7\textwidth]{bst-minmax-predsucc-result.png}\\
    }
  }
\end{frame}

\begin{frame}
  \frametitle{search a node}
  \only<+>{
    {\centering
      \includegraphics[width=0.7\textwidth]{bst-search-node.png}\\
    }
  }
  \only<+>{
    {\centering
      \includegraphics[width=0.7\textwidth]{bst-search-node-result.png}\\
    }
  }
\end{frame}

\begin{frame}
  \frametitle{insert a node}
  \only<+>{
    {\centering
      \includegraphics[width=0.7\textwidth]{bst-insert-node.png}\\
    }
  }
  \only<+>{
    {\centering
      Q: what is the process of building a tree from the sequence 5, 3, 2, 6, 9, 8, and 7?\\
    }
  }
  \only<+>{
    {\centering
      \includegraphics[width=0.7\textwidth]{bst-insert-node-result.png}\\
    }
  }
\end{frame}

\begin{frame}
  \frametitle{delete a node}
  \only<+>{
    {\centering
      \includegraphics[height=0.85\textheight]{clrs3-fig12-4.png}\\
      CLRS3 Figure 12.4 on Page 297\\
    }
  }
  \only<+>{
    {\centering
      \includegraphics[width=0.7\textwidth]{bst-delete-node.png}\\
    }
  }
  \only<+>{
    {\centering
      Q: What is the process of deleting 15, 18, and 13 in order from the running example\\[3ex]
      \includegraphics[width=0.7\textwidth]{example.png}\\
    }
  }
  \only<+>{
    {\centering
      \includegraphics[width=0.7\textwidth]{bst-delete-node-result.png}\\
    }
  }
\end{frame}

\begin{frame}
  \relax
  {\centering
    Q \& A\\
  }
\end{frame}

\begin{frame}
  \frametitle{to try the code, follow the instruction}
  \only<+>{\centering
    \url{https://github.com/pw4ever/cl-algo-n-struct}
  }
\end{frame}

%%%%%%%%%%%%%%%%%%%%%%%%%%%

\begin{frame}
  \relax
  {\centering
    reference solution\\
  }
\end{frame}

%\section{exercises}

% \begin{frame}[label=Q1]
%   \frametitle{Exercise}

%   \begin{myexer}
%   \end{myexer}

%   {\centering
%     \myinternalxref{S1}{Solution}\\
%   }
% \end{frame}

% \begin{frame}[label=A1]
%   \frametitle{Solution to \myinternalxref{Q1}{Exercise}}
%   \begin{proof}
    
%   \end{proof}
% \end{frame}

\begin{frame}[label=A2]
  \frametitle{Solution to \myinternalxref{Q2}{Exercise}}
  \begin{proof}
    (hint) later-inserted elements go \alert{down} along the tree hierarchy. therefore, parents are inserted earlier than children.

    \alert{``an''} rather than \alert{``the''} order: sibling insertion order can be arbitrary

    \alert{a} solution: pre-order walk
  \end{proof}
\end{frame}

\begin{frame}[label=A1]
  \frametitle{Solution to \myinternalxref{Q1}{Exercise}}
  \begin{proof}
    let us see.\\[4ex]
    {\centering
      \includegraphics[width=0.7\textwidth]{bst-walk-result.png}\\
    }
  \end{proof}
\end{frame}




\end{document}
